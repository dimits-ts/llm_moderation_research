% !TEX root = ../main.tex
%
\chapter{Limitations}
\label{sec:limitations}

A significant constraint in our research was the use of a relatively small LLM, driven by resource limitations. The model had also been significantly outdated at the time of this publication. The latter was made especially evident in later experiments using the \code{llama-3-7b} model, which produced more faithful discussions (outside this thesis), despite its significantly smaller size. These factors constrained both the quality of the synthetic discussions and how many experiments we were able to run in the duration of this thesis. 

Additionally, as explained in Section \ref{sec:related}, a significant limitation in this thesis was the absence of reliable, computational measures for not only argument quality, but also faithfulness to human speech. This constricted our ability to present decisive evidence for, or against, many of the Research Questions posed in Section \ref{sec:intro}.

Finally, while promising, our proposed statistical test can be significantly improved by modifying the null hypothesis ($H_0$) to check for \textit{any}, as opposed to complete reduction in polarization between all annotations and the annotations split by feature. 

We thus expect that addressing the issues in our approach, as well as using larger, modern models, would improve outcomes, both in generating and annotating discussions with \ac{SDB} prompts.