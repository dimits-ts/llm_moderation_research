% !TEX root = ../main.tex
%
\pdfbookmark[0]{Abstract}{Abstract}

% do not include a blank page between these two chapters
\let\cleardoublepage\clearpage

\chapter*{Abstract}
\label{sec:abstract}

Online discussion facilitation is crucial for discussions to flourish and prevent polarization and toxicity, which seem omnipresent. However, by being heavily based on humans, this facilitation proves costly, time-consuming and non-scalable, which has led many to turn to LLMs for discourse facilitation. In this thesis, we explore the use of LLMs as pseudo-users in online discussions, as a cost-efficient, realistic and scalable way to substitute initial LLM facilitation experiments, which would ordinarily necessitate costly human involvement. Furthermore, we show that including socio-demographic backgrounds in our LLM users leads to more realistic discussions. We explore the use of LLM annotators to estimate discussion quality, although we prove that using socio-demographic backgrounds in LLM annotators does not meaningfully affect their judgments. Finally, we release a synthetic-discussion creation and annotation framework, three synthetic datasets resulting from our experiments as well as subsequent analysis and findings from these datasets.\footnote{\url{https://github.com/dimits-ts/llm_moderation_research}}

\chapter*{Περίληψη}
\label{sec:abstract_greek}


Οι διαδικτυακοί χώροι συζήτησης έχουν καταστεί ζωτικής σημασίας για τον υγιή διάλογο μεταξύ δισεκατομμυρίων ανθρώπων και για πολλές δημοκρατικές διαδικασίες. Ωστόσο, μαστίζονται από την τοξικότητα και την πόλωση. Οι σύγχρονες τεχνικές συντονισμού/διαμεσολάβησης (moderation/facilitation) διαλόγου είναι αποτελεσματικές στη βελτίωση της ποιότητας των συζητήσεων, αλλά απαιτούν ανθρώπινη συμμετοχή και, ως εκ τούτου, είναι δαπανηρές και μη επεκτάσιμες. Τα Μεγάλα Γλωσσικά Μοντέλα (ΜΓΜ, ή LLMs στα αγγλικά) μπορούν να παρακάμψουν αυτά τα προβλήματα αντικαθιστώντας εν μέρη τους ανθρώπινους διαμεσολαβητές, αλλά η ανάπτυξη συνθετικών διαμεσολαβητών είναι αργή, επιρρεπής σε σφάλματα και συνήθως απαιτεί ανθρώπινη συμμετοχή σε πειράματα, αυξάνοντας σημαντικά το κόστος της. 

Στα πλαίσια της διατριβής αυτής, δημιουργούμε ένα νέο σύστημα το οποίο παράγει συνθετικές διαδικτυακές συζητήσεις χρησιμοποιώντας ψευτο-χρήστες ΜΓΜ με κοινωνικο-δημογραφικά υπόβαθρα έτσι ώστε να καταστήσουμε τις συζητήσεις ρεαλιστικές. Επεκτείνουμε το σύστημα μας με τη δυνατότητα υποστήριξης αυτόματων επισημειωτών (με χρήση ΜΓΜ), για την αντιμετώπιση του προβλήματος της αξιολόγησης διαλόγων. Οι ψευτο-επισημειωτές αυτοί έχουν προκαθορισμένα από εμάς κοινωνικο-δημογραφικά υπόβαθρα, έτσι ώστε να εκτιμήσουμε τη διαφωνία που πιθανώς να υπάρχει ανάμεσα σε ανθρώπους με αντίστοιχα υπόβαθρα. Τέλος, αναλύουμε την επίδραση διαφόρων παραγόντων στην τοξικότητα των συνθετικών συζητήσεων, ως υποκατάστατη μετρική της ποιότητάς τους. 

Δίνουμε δημόσια τον πηγαίο κώδικα του συστήματος, τρία συνθετικά σύνολα δεδομένων που αφορούν τις ίδιες τις συνθετικές συζητήσεις, τις επισημειώσεις τους, και τα αμφιλεγόμενα σχόλια σύμφωνα με τους διάφορους επισημειωτές ΜΓΜ. Η διατριβή εμπεριέχει επίσης και πειράματα, γραφήματα και στατιστικούς ελέγχους που αποδεικνύουν τα συμπεράσματά μας. Συμπεραίνουμε ότι \textit{οι συνθετικές συζητήσεις που διεξάγονται αποκλειστικά μέσω χρηστών ΜΓΜ, μπορούν να βοηθήσουν στον εντοπισμό μοτίβων συμπεριφοράς ανάλογα με την επιλεγμένη τεχνική διαμεσολάβησης}. Από την άλλη, \textit{διαψεύδουμε} την επίδραση των κοινωνικο-δημογραφικών υποβάθρων στην επισημείωση δεδομένων με ΜΓΜ.

