% !TEX root = ../main.tex
%
\pdfbookmark[0]{Abstract}{Abstract}

\chapter*{Abstract}
\label{sec:abstract}

Online discussion facilitation is crucial for discussions to flourish and prevent polarization and toxicity, which seem omnipresent. However, since it relies on humans, it proves costly, time-consuming and non-scalable, which has led many to turn to LLMs for discourse facilitation. In this thesis, we demonstrate that using LLMs as pseudo-users in online discussions can be a cost-efficient, realistic and scalable way to substitute initial LLM facilitation experiments which would ordinarily necessitate costly human involvement. Furthermore, we show that including socio-demographic backgrounds in our LLM users leads to more realistic discussions. We explore the use of LLM annotators to estimate discussion quality, although we prove that using socio-demographic backgrounds in LLM annotators does not meaningfully affect their judgements. Finally, we release a synthetic-discussion creation and annotation framework, three synthetic datasets resulting from our experiments as well as subsequent analysis and findings from these datasets\footnote{\url{https://github.com/dimits-ts/llm_moderation_research}}.

\chapter*{Περίληψη}
\label{sec:abstract_greek}


Οι διαδικτυακοί χώροι συζήτησης έχουν καταστεί ζωτικής σημασίας για τον υγιή διάλογο μεταξύ δισεκατομμυρίων ανθρώπων και για πολλές δημοκρατικές διαδικασίες. Ωστόσο, μαστίζονται από την τοξικότητα και την πόλωση. Οι σύγχρονες τεχνικές συντονισμού/διαμεσολάβησης (moderation/facilitation) διαλόγου είναι αποτελεσματικές στη βελτίωση της ποιότητας των συζητήσεων, αλλά απαιτούν ανθρώπινη συμμετοχή και, ως εκ τούτου, είναι δαπανηρές και μη επεκτάσιμες. Τα Μεγάλα Γλωσσικά Μοντέλα (ΜΓΜ, αγγλικά LLMs) μπορούν να παρακάμψουν αυτά τα προβλήματα αντικαθιστώντας εν μέρη τους ανθρώπινους διαμεσολαβητές, αλλά η ανάπτυξη συνθετικών διαμεσολαβητών είναι αργή, επιρρεπής σε σφάλματα και συνήθως απαιτεί ανθρώπινη συμμετοχή σε πειράματα, αυξάνοντας σημαντικά το κόστος της. 

Στα πλαίσια της διατριβής αυτής, δημιουργούμε ένα νέο σύστημα το οποίο παράγει συνθετικές διαδικτυακές συζητήσεις χρησιμοποιώντας ψευτο-χρήστες ΜΓΜ με κοινωνικο-δημογραφικά υπόβαθρα έτσι ώστε να καταστήσουμε τις συζητήσεις ρεαλιστικές. Επιλύουμε το έργο της επισημείωσης της ποιότητας των συζητήσεων, επεκτείνοντας το σύστημα μας με τη δυνατότητα υποστήριξης επισημείωτων ΜΓΜ. Οι ψευτο-επισημειωτές αυτοί έχουν δικά τους κοινωνικο-δημογραφικά υπόβαθρα έτσι ώστε να μοντελοποίησουμε τη διαφωνία των ανθρώπινων αντιστοίχων τους, με βάση τις προκαταλήψεις τους. Τέλος, αναλύουμε την επίδραση διαφόρων παραγόντων στην τοξικότητα των συνθετική συζητήσεων, ως υποκατάστατη μετρική της ποιότητάς τους. 

Δίνουμε στη δημοσιότητα τον πηγαίο κώδικα του συστήματος, τρία συνθετικά σύνολα δεδομένων που αφορούν τις ίδιες τις συνθετικές συζητήσεις, τις επισημειώσεις τους και τα αμφιλεγόμενα σχόλια σύμφωνα με τους διάφορους επισημειωτές ΜΓΜ, καθώς και πειράματα, γραφήματα και στατιστικούς ελέγχους που αποδεικνύουν τα συμπεράσματά μας. Συμπεραίνουμε ότι \textit{οι συνθετικές συζητήσεις που διεξάγονται αποκλειστικά μέσω χρηστών ΜΓΜ, μπορούν να βοηθήσουν στον εντοπισμό μοτίβων συμπεριφοράς ανάλογα με τον τροπο διαμεσολάβησης}. Από την άλλη, διαψεύδουμε την επίδραση των κοινωνικο-δημογραφικών υποβάθρων στην επισημείωση δεδομένων.

