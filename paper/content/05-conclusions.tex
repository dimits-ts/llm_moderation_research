% !TEX root = ../main.tex
%
\chapter{Conclusions}
\label{sec:conclusions}

In this thesis, we researched the practical feasibility of LLM generation for synthetic online discussions. We created a custom framework supporting automated synthetic discussion, annotation and analysis, and explored two different prompting strategies; vanilla instruction prompting as well as framing the discussion as a competitive, scorable game. We then used this framework to generate three synthetic datasets, containing discussions, annotations by LLM annotators with different \acp{SDB}, and controversial comments respectively. 

In the context of this research, we used toxicity as a proxy for argument quality. Analyzing the synthetic dataset, we found that the presence of a moderator can be a decisive influence on the toxicity of a discussion. Furthermore, framing the discussion as a scorable game seems to potentially keep LLM users in line using the threat of a moderator whose presence may not be perceivable. Finally, we proved that using different SDBs in LLM annotators yields no significant qualitative difference, and that any difference can be attributed to a change in priors, as opposed to reacting differently according to the content and context of the synthetic messages.

