% !TEX root = ../main.tex
%
\chapter{Conclusions \& Future Work}
\label{sec:conclusions}

In this thesis, we researched the practical feasibility of LLM generation for synthetic online discussions. We created a custom framework supporting automated synthetic discussion, annotation and analysis, and explored two different prompting strategies; vanilla instruction prompting as well as framing the discussion as a competitive, scorable game. We then used this framework to generate three synthetic datasets, containing discussions, annotations by LLM annotators with different \acp{SDB}, and controversial comments respectively. 

In the context of this research, we used toxicity as a proxy for argument quality. Analyzing the synthetic dataset, we found that the presence of a moderator can be a decisive influence on the toxicity of a discussion. Furthermore, framing the discussion as a scorable game seems to potentially keep LLM users in line using the threat of a moderator whose presence may not be perceivable. Finally, we proved that using different \acp{SDB} in LLM annotators yields no significant qualitative difference, and that any difference can be attributed to a change in priors, as opposed to reacting differently according to the content and context of the synthetic messages.

Future work should expand on making synthetic conversations more realistic, ideally rendering them indistinguishable from human online conversations. Additionally, there is room for experimentation involving scaling-up the number of \acp{SDB} and the information involved in them (age, education level, country of origin etc.). Furthermore, the \ac{SDF} enables the possibility of large-scale experiments exploring the effects of different moderating techniques, interventions and model instances on conversation quality. Finally, the findings of the synthetic experiments should be replicated with human participants, both to achieve concrete results on LLM facilitation, and verify the applicability of synthetic experiments themselves to real world experimentation with humans.

