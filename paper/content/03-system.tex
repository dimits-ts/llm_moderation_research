% !TEX root = ../main.tex
%
\chapter{System Design and Implementation}
\label{sec:system}

A very important part of this thesis is the Synthetic Discussion Framework (SDF), a lightweight, specialized python library which supports the automatic creation of dialogues through LLMs. In this section we explain in detail the initial requirements for this framework, why commercially-available alternatives do not fit these requirements (Section \ref{sec:system:requirements}), the general design and concept (Section \ref{sec:system:design}), and finally the actual implementation (Section \ref{sec:system:implementation}) of the new framework.

\section{Requirements}
\label{sec:system:requirements}

The requirements for the SDF were not obtained by standard Requirement Solicitation procedures. Instead, they were iteratively solicited during weekly meetings. Thus, no formal document detailing them exists. 

However, the interested stakeholders in the context of the wider research effort, ultimately decided on a combination of the below requirements. We denote the SDF as "the system" for this section.

Functional requirements:
\begin{enumerate}
	\item The system must support multiple LLM types, with potentially different libraries handling them.
	\item The system must support a conversation with at least two LLM users.
	\item The system must support socio-demographic backgrounds (SDBs) to be given to LLM users.
	\item The system must support the existence and absence of a third LLM user, posing as a moderator.
	\item The moderator must be able to intervene at any point in the conversation.
	\item The moderator must be able to "ban" users, preventing them from further commenting.
	\item The output of the system must be serializable and easily parsable.
\end{enumerate}

Non functional requirements:
\begin{enumerate}
	\item The system must be able to be ran locally, with scarce computational resources.
	\item The system must be accessed through a simple and flexible API.
	\item The system must be able to automatically produce a large amount of synthetic discussions in a timeframe of hours.
\end{enumerate}

Current LLM discussion frameworks such as Concordia \cite{Vezhnevets2023GenerativeAM} and LangChain \cite{langchain} fit, or can be made to fit, all functional requirements listed above. They however fail at all three non-functional requirements, as they are industrial-grade frameworks, meant for a diverse set of business use-cases making, their API convoluted. Of course, this could be circumvented by employing the Adapter pattern \cite{gamma1995design}. The problem then would be that their internal components frequently necessitate computer resources (dedicated RAM, GPU VRAM e.t.c.) which, for a smaller application such as ours, will most likely not be used to their fullest.

Thus the solution of building our own framework is the only practical way of satisfying all the requirements above.



\section{Design}
\label{sec:system:design}



\section{Implementation}
\label{sec:system:implementation}

