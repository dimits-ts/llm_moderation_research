% !TEX root = ../main.tex
%
\pdfbookmark[0]{Abstract}{Abstract}

% do not include a blank page between these two chapters
\let\cleardoublepage\clearpage

\chapter*{Abstract}
\label{sec:abstract}

Online discussion moderation/facilitation is crucial for discussions to flourish and prevent polarization and toxicity, which nowadays seems omnipresent. However, because it is heavily based on humans, moderation/facilitation proves costly, time-consuming and non-scalable, which has led many to turn to LLMs for discourse facilitation. In this thesis, we explore the use of LLMs as pseudo-users in online discussions, as a cost-efficient, realistic and scalable way of substituting initial LLM facilitation experiments, which would ordinarily necessitate costly human involvement. Furthermore, we show that including socio-demographic backgrounds in our LLM users leads to more realistic discussions. We explore the use of LLM annotators to estimate discussion quality, using a new statistical test to gauge annotator polarization, and show that using socio-demographic backgrounds in LLM annotators may not meaningfully affect their judgments. Finally, we release a synthetic-discussion creation and annotation framework, the synthetic datasets resulting from our experiments, as well as subsequent analysis and findings from these datasets. Code, datasets and analysis can be found at \url{https://github.com/dimits-ts/llm_moderation_research}.

\textWarning{Content Warning: This paper contains samples of harmful text, including violent, toxic, controversial, and potentially illegal statements.}

\chapter*{Περίληψη}
\label{sec:abstract_greek}

% greek hyphenation does not work
\begin{hyphenrules}{nohyphenation}
	\sloppy
	
	Ο συντονισμός/διαμεσολάβση (moderation/facilitation) των διαδικτυακών συζητήσεων είναι ζωτικής σημασίας για την άνθηση των συζητήσεων και την αποτροπή της πόλωσης και της τοξικότητας, που στις μέρες μας φαίνεται πανταχού παρούσες. Οι σύγχρονες τεχνικές συντονισμού/διαμεσολάβησης απαιτούν ανθρώπινη συμμετοχή και, ως εκ τούτου, είναι δαπανηρές και μη επεκτάσιμες, οδηγώντας πολλούς να στραφούν στη χρήση Μεγάλων Γλωσσικών Μοντέλων (ΜΓΜ, ή LLMs στα Αγγλικά) για αυτές. Στα πλαίσια της διατριβής αυτής δημιουργούμε ένα νέο σύστημα το οποίο παράγει συνθετικές διαδικτυακές συζητήσεις, χρησιμοποιώντας ψευτο-χρήστες ΜΓΜ με κοινωνικο-δημογραφικά υπόβαθρα έτσι ώστε να καταστήσουμε τις συζητήσεις ρεαλιστικές. Επιπλέον, δείχνουμε ότι η χρήση κοινωνικο-δημογραφικών υποβάθρων οδηγεί σε πιο ρεαλιστικές συζητήσεις. Διερευνούμε τη χρήση των σχολιαστών LLM για την εκτίμηση της ποιότητας των συζητήσεων, χρησιμοποιώντας ένα νέο στατιστικό έλεγχο για τη μέτρηση της πόλωσης των σχολιαστών και δείχνουμε ότι η χρήση κοινωνικο-δημογραφικού υπόβαθρου στους σχολιαστές LLM μπορεί να μην επηρεάζει σημαντικά τις κρίσεις τους. Επεκτείνουμε το σύστημα μας με τη δυνατότητα υποστήριξης αυτόματων επισημειωτών (με χρήση ΜΓΜ), για την αντιμετώπιση του προβλήματος της αξιολόγησης διαλόγων. Οι ψευτο-επισημειωτές αυτοί έχουν προκαθορισμένα από εμάς κοινωνικο-δημογραφικά υπόβαθρα, έτσι ώστε να προσομοιώσουμε τη διαφωνία που πιθανώς να υπάρχει ανάμεσα σε ανθρώπους με αντίστοιχα υπόβαθρα. Τέλος, δίνουμε στη δημοσιότητα το δικό μας πρόγραμμα δημιουργίας και σχολιασμού συνθετικών συζητήσεων, τα συνθετικά σύνολα δεδομένων που προέκυψαν από τα πειράματά μας, καθώς και την επακόλουθη ανάλυση και τα συμπεράσματα από αυτά. Ο κώδικας, τα σύνολα δεδομένων και η ανάλυση βρίσκονται στο αποθετήριο κώδικα στη διεύθυνση \url{https://github.com/dimits-ts/llm_moderation_research}.
	
	\fussy
\end{hyphenrules}